
\section{爱好}
\subsection{听力句子}
\kewen{
  \item[]\ruby{音楽}{おん|がく}は聞きません
  \item[]\ruby[g]{週末}{しゅうまつ}は音楽を聞きます
  \item[]二人で音楽を聞きます
  \item[]音楽は聞きません
  \item[]彼はよく\ruby{本}{ほん}を読みます
  \item[]本は\ruby[g]{あまり}{怎么}読みません
  \item[]\ruby[g]{きのう}{昨日}は歌を習いました
}

\subsection{日中互译}
\jc{おとといはぜんぜん\ruby{歌い}{うた|い}ませんでした。}{前天完全没有唱歌。}
\cj{两个人听音乐}{\ruby{音楽}{おん|がく}を二人で\ruby{聞}{き}きます}
\cj{我喜欢这首歌}{私はこの歌が好きです}
\jc{歌が好きです}{喜欢歌曲}
\cj{周末听音乐}{週末は音楽を聞きます}
\cj{喜欢歌曲}{歌が好きです}
\jc{歌がきらいです}{讨厌歌曲}
\jc{\ruby[g]{時々}{ときどき}一人で旅行します}{有时一个人旅行}
\cj{不怎么旅行}{旅行はあまりしません}
\jc{彼女はあまり走りません}{她不怎么跑步}
\cj{那些小学生经常跑步}{その小学生たらはよく走ります}
\jc{\ruby{歌}{うた}は歌いません}{不唱歌}
\jc{きのうはたくさん歌いました}{昨天唱了很多歌}
\jc{本をたくさん読みます}{看很多书}
\cj{他不喜欢画}{彼はえが好きじゃないです}
\jc{何のえを描きますか}{画什么画?}
\cj{星期一学习中文}{月曜日は中国語を習います}
\jc{何を習いますか}{学习什么?}
\jc{新しい本を書きましたか}{写新书了吗?}
\jc{新しい本を書きました}{写新书了。}
\jc{昨日は高橋先生を見ましたか?}{昨天看见高桥老师了吗?}
\cj{铃木先生写日语书了}{鈴木さんは日本語の本を書きました。}


\subsection{生詞表}
\word{走り[はしり]}{[动]跑步}
\word{旅行[りょこう]}{[动]旅行}
\word{歌い[うたい]}{[动]唱}
\word{音楽[おんがぐ]}{[名]音乐}
\word{聞き[きき]}{[动]听}
\word{映画[えいが]}{[名]电影}
\word{習い[ならい]}{[动]学习}
\word{読み[よみ]}{[动]看}
\word{描き[えがき]}{[动]画}
\word{見[み]}{[动]看;看见}
\word{絵[え]}{[名]画}
\word{書き[かき]}{[动]写}
\section{爱好}

\subsection{听力句子}
\kewen{
  \item[] hello here
}
\subsection{日中互译}
\jc{おとといはぜんぜん
uby{歌い}{うた|い}ませんでした。}{前天完全没有唱歌。}
\subsection{生詞表}
\word{走り[はしり]}{[动]跑步}
